\documentclass{article}

\usepackage{a4wide}
\usepackage{tabularx}
\title{AS8051 User Guide}
\author{Luk Bettale}
\date{}

\begin{document}

\maketitle

\section*{Introduction}
AS8051 is a 2 passes assembler for 8051 assembly language. In addition
to the classical 8051 assembly, this assembler supports some
directives and macro instructions that are detailed hereafter.

\section*{Using \texttt{as8051}}
AS8051 can be launched by command line. The assembly input file has to
be specified as first argument. The \texttt{-b} option instructs the
ASM to output in plain binary format instead of HEX format.

\begin{verbatim}
as8051 [-b] input
\end{verbatim}

\section*{Assembler Macro Instructions}
Two macro instructions are supported: \texttt{jmp} and
\texttt{call}. When using \texttt{jmp} (resp. \texttt{call}), if the
label is a backward reference, these macros expands to the most
suitable between \texttt{sjmp}, \texttt{ajmp}, \texttt{ljmp}
(resp. \texttt{acall}, \texttt{lcall}). If the reference is forward,
expands to \texttt{ljmp} (resp. \texttt{lcall}).

\section*{Assembler directives}
The following directives are supported:

\noindent
\begin{tabularx}{\textwidth}{rX}
  \texttt{SEGMENT}&Declares a new segment in one of the 4 memory
  locations (CODE, DATA, XDATA, BIT)\\
  &\texttt{globvars SEGMENT XDATA}\\
  \noalign{\smallskip}
  \hline
  \noalign{\smallskip}
  \texttt{RSEG}&Effectively use a previously declared segment (for the
  following lines until another segment change)\\
  &\texttt{\ \ \ \ RSEG globvars}\\
  \noalign{\smallskip}
  \hline
  \noalign{\smallskip}
  \texttt{ORG}&Specifies the location of the following lines in
  code. This is used to place code at specific locations.\\
  &\texttt{\ \ \ \ ORG 0x0100 ;; following code starts at address 0x0100}\\
  \noalign{\smallskip}
  \hline
  \noalign{\smallskip}
  \texttt{DSEG}&Switch to data segment (first available)\\
  &\texttt{\ \ \ \ DSEG}\\
  \noalign{\smallskip}
  \hline
  \noalign{\smallskip}
  \texttt{XSEG}&Switch to xdata segment (first available)\\
  &\texttt{\ \ \ \ XSEG}\\
  \noalign{\smallskip}
  \hline
  \noalign{\smallskip}
  \texttt{BSEG}&Switch to bit addressable data segment (first available)\\
  &\texttt{\ \ \ \ BSEG}\\
  \noalign{\smallskip}
  \hline
  \noalign{\smallskip}
  \texttt{CSEG}&Switch to code segment (first available)\\
  &\texttt{\ \ \ \ CSEG}\\
  \noalign{\smallskip}
  \hline
  \noalign{\smallskip}
  \texttt{AT}&Used with one of the 4 previous directives, specifies
  the location of the segment instead of switching to the first
  available. Note that \texttt{CSEG AT} becomes equivalent to \texttt{ORG}.\\
  &\texttt{\ \ \ \ XSEG AT 0x0080}\\
  \noalign{\smallskip}
  \hline
  \noalign{\smallskip}
  \texttt{DS}&Reserve some space in the current segment (not initialized).\\
  &\texttt{\ \ \ \ DS 8 ; reserve 8 bytes}\\
  \noalign{\smallskip}
  \hline
  \noalign{\smallskip}
  \texttt{DBIT}&Reserve some bits in the current segment (only for bit
  addressable data segment).\\
  &\texttt{\ \ \ \ DBIT 5 ; reserve 5 bits}\\
  \noalign{\smallskip}
  \hline
  \noalign{\smallskip}
  \texttt{EQU}&Declares a constant.\\
  &\texttt{myconst EQU 38+4 ; myconst expands to 42}\\
  \noalign{\smallskip}
  \hline
  \noalign{\smallskip}
  \texttt{SET}&Declares a value that can change within the file.\\
  &\texttt{myvar SET 4 ; myvar expands to 4}\\
  &\texttt{myvar SET myvar+1 ; now myvar expands to 5}\\
  \noalign{\smallskip}
  \hline
  \noalign{\smallskip}
  \texttt{USING}&Change register bank for expansion of the ARi\\
  &\texttt{\ \ \ \ USING 3 ; now AR0 equals to 0x18 (third register bank)}\\
  \noalign{\smallskip}
  \hline
  \noalign{\smallskip}
  \texttt{END}&Marks the end of the file. What follows is not considered.\\
  &\texttt{\ \ \ \ END}\\
  \noalign{\smallskip}
  \hline
\end{tabularx}

\end{document}
